\specsection{1.9 Сформулировать определение схемы испытаний Бернулли. Доказать формулу для вычисления вероятности реализации ровно k успехов в серии из n испытаний по схеме Бернулли. Доказать следствия этой формулы}

\OPR Схемой Бернулли (биномиальной схемой) наз. серию экспериментов указанного вида, которая обладает

\begin{enumerate}[topsep=0pt, leftmargin=20pt, noitemsep]
	\item Все испытания независимы, т.е. исход $k$-го испытания не зависит от остальных
	\item Вероятность наступления успеха во всех испытаниях неизменна
\end{enumerate}

\THRM $P_n(k)=C_n^kp^kq^{n-k},~k=0,1,...,n$

\B{Доказательство}

\begin{enumerate}[topsep=0pt, leftmargin=20pt, noitemsep]
	\item Результат серии из $n$ испытаний будем описывать кортежем $\omega = (x_1,...,x_n)$, где $x_i =
	\begin{cases}
		1,  & \text{если в } i \text{ испытании удача}\\
		0,  & \text{иначе}
	\end{cases}$

	\item $A$ = \{из $n$ испытаний произошло ровно $k$ успехов\}
	\item [] Тогда $A=\{\omega : \text{ровно k единиц}\}$. Число исходов в $A$ равно количеству способов поставить в кортеже $\omega$ ровно $k$ единиц = числу способов выбрать в $\omega~k$ позиций для расстановки единиц = $C_n^k$
	
	\item Для каждого $\omega = (x_1...x_n)\in A$
	\item [] $P(\omega)=P(x_1...x_n)=P(\{\text{в 1 испытании результат }x_1\})\cdot...\cdot P(\{\text{в n испыт. рез. }x_n\}) = $
	\item [] $= |\text{ровно k успехов и n неудач}| = p^kq^{n-k}$
	
	\item т.к. $|A|=C_n^k$, то $P(A)=C_n^kp^kq^{n-k}$ 
\end{enumerate}

\SLED 1: Вероятность того, что число успехов в серии из $n$ испытаний по схеме Бернулли не менее $k_1$ и не более $k_2$: $P(k_1\leq k\leq k_2)=\suml_{i=k_1}^{k_2}C_n^ip^iq^{n-i},~k_1\leq k_2$

\B{Доказательство} Пусть $A$= \{произошло $\geq k_1$ и $\leq k_2$ успехов\}

Тогда $A=A_{k_1}+...+A_{k_2}$, где $A_i$=\{произошло ровно $i$ успехов\}, $i$=$\overline{k_1,k_2}$

$P(A)=P(\suml_{i=k_1}^{k_2}A_i) = |A_i \text{ несовместны }| = \suml_{i=k_1}^{k_2}C_n^ip^iq^{n-i}$\newline

\SLED 2: Вероятность того, что в серии из $n$ испытаний по схеме Бернулли произойдёт хотя бы один успех можно посчитать по формуле $P_n(k\geq 1)=1-q^n$

\B{Доказательство} 

$P_n(k\geq 1)=1-P(\{\text{в серии из n испыт. будет 0 успехов}\})=1-\underbrace{P_n(0)}_{C_n^0p^0q^{n-0}}=1-q^n$

\clearpage
