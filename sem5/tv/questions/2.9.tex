\specsection{2.9 Понятие функции скалярной случайной величины. Доказать теорему о формуле для вычисления плотности $f_Y(y)$ случайной величины $Y=\varphi(X)$, если $X$ -- непрерывная случайная величина, а $\varphi$ -- монотонная непрерывно дифференцируемая функция. Записать аналогичную формулу для
кусочно-монотонной функции $\varphi$}

Скалярная функция скалярного аргумента:

Пусть $X$ -- некоторая сл. вел., $\varphi:\R\rightarrow\R$ -- некоторая известная ф-и. 

Тогда $\varphi(X)=Y$ -- некоторая сл. вел

\THRM Пусть
\begin{enumerate}[topsep=0pt, leftmargin=20pt, noitemsep]
	\item $X$ -- непрерывная случайная величина
	
	\item $f_X(x)$ -- плотность распределения сл. вел.
	
	\item $\varphi:\R\rightarrow\R$ -- монотонная ф-ия, которая непрер. дифф.
	
	\item $\psi$ -- ф-ия, обратная к $\varphi$
	
	\item $Y=\varphi(X)$
	
	\item [] Тогда
	\setcounter{enumi}{0}
	
	\item $Y$ -- также является непрерывной сл. вел.

	\item $f_Y(y)=f_X(\psi(y))|\psi'(y)|$
	
	\item [] Доказательство
	\setcounter{enumi}{0}
	
	\item [] По опр. функции распределения $F_Y(y)=P\{Y<y\}=P\{\varphi(X)<y\}$
	\item [] Т.к. $\varphi$ монотонная, то существует обратная к ней ф-ия $\varphi^{-1}=\psi$
	\item []$
	\begin{cases}
		\text{монотоннно}\uparrow,  & F_Y(y)=P\{X<\psi(y)\}=F_X(\psi(y))\\
		\text{монотоннно}\downarrow,  & F_Y(y)=P\{X>\psi(y)\}=1-P\{X\leq\psi(y)\}\stackrel{\text{X-непрер.}}{=}\\
		~ & = 1-P\{X<\psi(y)\}=1-F_X(\psi(y))
	\end{cases}$

	\item []$
	\begin{cases}
		\text{монотоннно}\uparrow,  & \tfrac{d}{dy}[F_X(\psi(y))]=F'_X(\psi(y))\cdot\psi'(y)\\
		\text{монотоннно}\downarrow,  & \tfrac{d}{dy}[1-F_X(\psi(y))]=-F'_X(\psi(y))\cdot\psi'(y)=f_X(\psi(y))\cdot|\psi'(y)|
	\end{cases}$
		
\end{enumerate}

~

\THRM Пусть
\begin{enumerate}[topsep=0pt, leftmargin=20pt, noitemsep]
	\item $X$ -- непрерывная случайная величина
	
	\item $\varphi:\R\rightarrow\R$ -- кусочно монотонная ф-ия, имеющая $n$ интервалов
	
	\item $\varphi$ -- диффиринцируема
	
	\item Для данного $y\in\R,x_1=x_1(y),...,x_k=x_k(y)~(k\leq n)$ -- все решения уравнения $y=\varphi(x)$, принадл. инт. $I_1,...,I_k$. Тогда: $f_Y(y)=\suml_{i=1}^k f_X(\psi_i(y))\cdot|\psi'_i(y)|$, где
	\item [] $\psi_i(y)$ -- ф-ия, обратная к $\varphi(x)$ на интервале $I_j,j=\overline{1,k}$
\end{enumerate}

\clearpage
