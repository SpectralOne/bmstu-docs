\specsection{2.12 Сформулировать определение дисперсии случайной величины. Механический смысл дисперсии.
Доказать свойства дисперсии. Понятие среднеквадратичного отклонения случайной величины.}

Пусть $X$ -- сл. вел., $m$ -- $MX$

\OPR Дисперсией случайной величины
$X$ называют число $DX=M[(X-m)^2]$

\ZAM 1. Дискр: $DX=|DX=M[(X-m)^2],\varphi(x)=(x-m)^2|=\suml_{i}(x_i-m)^2p_i$

\ZAM 2. Непрерывная: $DX=\intl_{-\infty}^{+\infty}(x-m)^2f(x)dx$, где $f$ -- ф-ия плотности сл. вел. $X$

Механический смысл: Дисперсия сл. вел. характеризует разброс значений этой сл. вел. относительно мат. ожидания. Чем больше дисперсия,
тем больше разброс.
С точки зрения механики дисперсия — момент инерции вероятностной массы
отн. мат. ожидания

Свойства дисперсии:
\begin{enumerate}[topsep=0pt, leftmargin=20pt, noitemsep, label=\arabic*\degree]
	\item $DX\geq 0$
	
	\item Если $P\{X=x_0\}=1$, то $DX=0$
	
	\item $D[aX+b]=a^2DX$
		
	\item $DX=M[X^2]-(MX)^2$
	
	\item [] Доказательства
	\setcounter{enumi}{0}
	
	\item $DX=MY$, где $Y=(X-m)^2\geq 0\Rightarrow MY\geq 0$
	
	\item $DX=|MX=x_0|=|\suml_{i}(x_i-m)^2p_i|=(x_0-x_0)^2\cdot 1=0$
	\begin{tabular}{ |c|c| } 
		\hline
		$X$ & $x_0$ \\ \hline
		$P$ & 1 \\ \hline
	\end{tabular}
	
	\item Обозначим $m=MX$
	\item [] $D[aX+b]=M[((aX+b) - M(aX+b))^2]=M[(aX+b-aM[X]-b)^2]=$
	\item [] $=M[a^2(X-MX)^2]=a^2M[(X-m)^2]=a^2DX$
	
	\item Обозначим $m=MX$
	\item [] $DX=M[(X-m)^2]=M[X^2-2XM+m^2]=M[X^2]-2mM[X]+M[m^2]=$
	\item [] $=M[X^2]-m^2=M[X^2]-(MX)^2$
	
	\item Обозначим $m_1=MX_1,m_2=MX_2$
	\item [] $D[X_1+X_2]=M[((x_1+x_2)-M(X_1+X_2))^2]=M[((x_1-m_1)+(x_2-m_2))^2]=$
	\item [] $=M[(X_1-m_1)^2+(X_2-m_2)^2+2(X_1-m_1)(X_2-m_2)]=$
	\item [] $=\underbrace{M[(X_1-m_1)^2]}_{DX_1}+\underbrace{M[(X_2-m_2)^2]}_{DX_2}+\underbrace{2M[(X_1-m_1)(X_2-m_2)]}_{A}=DX_1+DX_2$
	\item [] $A=M[X_1X_2-m_1X_2-m_2X_1+m_1m_2]=M[X_1X_2]-m_1MX_2-m_2NX_1+m_1m_2=$
	\item [] $=|x_1,x_2 - \text{независ.}\Rightarrow M(X_1,X_2)=m_1m_2|=m_1m_2-m_1m_2-m_1m_2+m_1m_2=0$

\end{enumerate}

\ZAM $DX$ имеет размерность, равную квадрату размерности случайной величины $X$. Это не всегда удобно, особенно при решении практических задач. Поэтому рассматривают такую числовую характеристику, как среднеквадратичное отклонение (СКО).

\OPR Среднеквадратичным отклонением (СКО) сл. вел. $X$ наз. число $\sigma_X=\sqrt{DX}$

\clearpage
