\specsection{1.7 Сформулировать определение пары независимых событий. Доказать критерий независимости
двух событий. Сформулировать определение попарно независимых событий и событий,
независимых в совокупности. Обосновать связь этих свойств}

Пусть $A$ и $B$ -- два события, которые связаны с некоторым случайным экспериментом. 

\OPR События $A$ и $B$ называют независимыми, если $P(AB)=P(A)P(B)$

\THRM 
\begin{enumerate}[topsep=0pt, leftmargin=20pt, noitemsep]
	\item Если $P(B) > 0$, то $A,B$ - независимы $\Leftrightarrow P(A|B) = P(A)$
	
	\item Если $P(A) > 0$, то $A,B$ - независимы $\Leftrightarrow P(B|A) = P(B)$
	
	\item [] Доказательство

	\setcounter{enumi}{0}
	
	\item $\Rightarrow$ Пусть $P(AB)=P(A)P(B)$
	\item [] тогда $P(A|B)=\tfrac{P(AB)}{P(B)}=\tfrac{P(A)P(B)}{P(B)} = P(A)$
	\item [] $\Leftarrow$ Пусть $P(A|B) = P(A)$
	\item [] тогда $P(A|B)=P(A)=\tfrac{P(AB)}{P(B)}\Rightarrow P(AB)= P(A)P(B)\Rightarrow A,B $ независимы
	
	\item аналогично 1
\end{enumerate}

\OPR События $A_1...A_n$ называются попарно независимыми, если $\forall i\neq j$ события $A_i, A_j$ = независимы, т.е. $P(A_iA_j)=P(A_i)P(A_j), i\neq j$

\OPR События $A_1...A_n$ наз. независимыми в совокупности, если для любого набора индексов $i_1...i_k\in \{1...n\}, k=\overline{1...n}$ справедливо $P(A_{i1}\cdot...\cdot A_{ik}) = P(A_{i1})\cdot...\cdot P(A_{ik})$

\ZAM Если $A_1...A_n$ независимы в совокупности, то они попарно независимы. Обратное неверно.

\PRIM Рассмотрим правильный тетраэдр, на одной грани которого написано '1', на другой грани -- '2', на третьей грани -- '3', а на четвёртой -- "1 2 3". Тетраэдр один раз подбрасывают. Пусть событие $A_1$ -- {На нижней грани написано '1'}. $A_2$ -- {На нижней грани написано '2'}, $A_3$ = {На нижней грани написано '3'}. 

Доказательство состоит в том, что события $A_1, A_2, A_3$ попарно независимы, но при этом не являются независимыми в совокупности.

$P(A_1) = P(A_2) = P(A_3) = \frac{1}{2}$.

$P(A_1 \cdot A_2) = P(A_1) \cdot P(A_2)$, и так для каждой пары.

Но $P(A_1 \cdot A_2 \cdot A_3) = \frac{1}{4} \neq P(A_1) \cdot P(A_2) \cdot P(A_3)$.

\clearpage
