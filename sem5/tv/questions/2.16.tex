\specsection{2.16 Понятие условного распределения компоненты двумерного случайного вектора (дискретный и непрерывный случаи). Сформулировать определения значений условного математического ожидания и условной дисперсии. Сформулировать определения условного математического ожидания и условной дисперсии. Записать формулы для вычисления условных математического ожидания и дисперсии для компоненты двумерного нормального вектора.}

Пусть $(X,Y)$ -- дискр. сл. век-р, $\pi_{ij}=P\{X=x_i|Y=y_j\}=\tfrac{P_{ij}}{P_{Yj}}$

\OPR Значением условного мат. ожидания сл. вел. $X$ при условии, что сл. вел. $X$ приняла значение $y_j$, наз. число $M[X|Y=y_j]=\suml_{i}X_i\pi_{ij}$

Если $(X,Y)$ -- непр. сл. век-р, то $f_X(x|Y=y)=\tfrac{f(x,y)}{f_Y(y)}$

\OPR Значением условного мат. ожидания сл. вел. $X$ при условии $Y=y$, наз. число $M[X|Y=y]=\intl_{-\infty}^{+\infty}xf_X(x|Y=y)dx$

Пусть $(X,Y)$ -- произвольный сл. век-р

\OPR Условным мат. ожиданием сл. вел. $X$ относительно сл. вел. $Y$ наз. ф-ия $g(Y)=M[X|Y]$ такая, что 

1) Область определения $g$ совпадает с мн-вом возможных значений сл. вел. $Y$

2) Для каждого возможного значения $y$ сл. вел. $Y~g(y)=M[X|Y=y]$

\ZAM Условное мат. ожидание явл. ф-ией сл. вел. $Y$ , т. е. оно само является сл. вел.

\ZAM Усл. мат. ожидание сл. вел. $Y$ относительно сл. вел. $X$ определяется аналогично

\OPR Условной дисперсией сл. вел. $X$ отн. сл. вел. $Y$ наз. сл. вел. 

$D[X|Y]=M[(X-M[X|Y])^2]$

\ZAM 1. Если $(X,Y)$ -- дискр. сл. век-р, то $D[X|Y=y_j]=\sum(x_i-M[X|Y=y_j])^2\pi_{ij}$

2. Если $(X,Y)$ -- непр. сл. век-р, то $D[X|Y=y_j]=\intl_{-\infty}^{+\infty}(x-M[X|Y=y])^2f_X(X|Y=y)dx$

Пусть $\overrightarrow{x}=(x_1,x_2)$ -- двумерный сл. вектор с $\overrightarrow{m}=(m_1,m_2)$ и $\sum = 
\left[\begin{array}{l}
	\sigma_1^2 ~~\rho\sigma_1\sigma_2 \\
	\rho\sigma_1\sigma_2 ~~\sigma_2^2
\end{array}\right]$

Тогда 1) условное распределение $X$ при $Y=y$ будет нормальным

2) $M[X|Y=y]=m_1+\rho\tfrac{\sigma_1}{\sigma_2}(y-m_2)$

$D[X|Y=y]=\sigma_1^2(1-\rho^2)$

\clearpage

