\specsection{1.8 Сформулировать определение полной группы событий. Доказать теорему (формулу) полной
вероятности и формулу Байеса. Понятия априорной и апостериорной вероятностей}

Пусть $\Omega$ -- пространство элементарных исходов, связанных с некоторым случайным экспериментом, а ($\Omega$, $\beta$, $P$) -- вероятностное пространство этого случайного эксперимента.

\OPR События $H_1...H_n$ образуют полную группу событий, если

1) $P(H_i) > 0$

2) $H_i\cdot H_j = \varnothing, ~ i\neq j$

3) $H_1+...+H_n=\Omega$

\ZAM При этом события $H_i,~i=\overline{1,n}$ часто наз. гипотезами

\THRM Формула полной вер-ти

Пусть

1) $H_1...H_n$ -- полная группа событий 

2) $P(H_i)>0,~i=\overline{1,n}$

Тогда $P(A)=P(A|H_1)P(H_1)+...+P(A|H_n)P(H_n)$ -- ф-ла полной вер-ти

Доказательство

1) $A = A\Omega = A(H_1+...+H_n) = AH_1+...+AH_n$

2) $P(A)=P(AH_1+...+AH_n)=\left|H_iH_j=\varnothing,~i\neq j\Rightarrow\underbrace{(AH_i)}_{H_i}\underbrace{(AH_j)}_{H_j}=\varnothing\right| =$

$= P(AH_1)+...+P(AH_n)\stackrel{\text{ф-ла умнож. вер-тей}}{=} P(A|H_1)P(H_1)+...+P(A|H_n)P(H_n)$\newline

\THRM Формула Байеса

Пусть

1) Выполнены все условия из \B{th}. о ф-ле полной вер-ти

2) $P(A) > 0$

Тогда $P(H_i|A)=\tfrac{P(A|H_i)P(H_i)}{P(A|H_1)P(H_1)+...+P(A|H_n)P(H_n)},~i=\overline{1,n}$\newline

\B{Доказательство}

$P(H_i|A)=\tfrac{P(AH_i)}{P(A)}\stackrel{\text{ф-ла полной вер-ти, th умнож вер-тей}}{=} \tfrac{P(A|H_i)P(H_i)}{P(A|H_1)P(H_1)+...+P(A|H_n)P(H_n)}$\newline

Вероятности $P(H_i),i=\overline{1,n}$ называются \B{априорными}, т. к. они известны до опыта

Вероятности $P(H_i|A),i=\overline{1,n}$ называются \B{апостериорными}, т.к. они вычисляются
после опыта.





\clearpage
