\specsection{1.2 Определение пространства элементарных исходов, примеры. Понятие события(нестрогое).
Сформулировать геометрическое и статистическое определения вероятности. Достоинства и
недостатки этих определений.}

\subsubsection*{1. Пространство событий. Понятия событий. Нестрогое событие.}

\OPR Случайным экспериментом называется эксперимент, результат которого невозможно заранее предсказать.

\OPR Каждый неделимый результат случ. экспер. называют элементарным исходом.

\OPR Мн-во всех элементарных исходов случайной величины $\Omega$ называется пространством элементарных исходов.

\PRIM
\begin{enumerate}[topsep=0pt, leftmargin=18pt, noitemsep]
	\item Бросаем монету. Возможные исходы: \texttt{O} или \texttt{P}.~~ $\Omega = \{\texttt{O},~\texttt{P}\}~~|\Omega|=2$
	
	\item Из колоды в 36 карт последовательно извлекают 2 карты.~~ 
	$\Omega=\{(x_1,x_2):~x_1,x_2\in \{1,\dots,36\},~x_1\neq x_2\}$, $x_i$ -- 
	номер карты при $i$-ом извлечении.~~$|\Omega|=36\cdot 35$
\end{enumerate}

В результате случайного эксперимента, проведённого однократно, обязательно реализуется один из элементарных исходов.

\OPR (нестрогое) Событием (случайным событием) в рамках данного случайного эксперимента называется любое подмножество пространства элементарных исходов $\Omega$ этого эксперимента.

При этом говорят, что в результате случайного эксперимента (СЭ) наступило событие А, если имел место один из входящих в А элементарных исходов.

\OPR Событие $B$ наз. следствием события $A$, если из того, что произошло $A$ следует, что произошло $B$. ~$B\subseteq A$

\ZAM Любое мн-во $\Omega$ содержит два подмн-ва: $\varnothing$, $\Omega$. Соотв. события называются \B{невозможными} ($\varnothing$) и \B{достоверными} ($\Omega$). Эти события наз. несобственными, остальные -- собственными.

\PRIM Из урны с 2 белыми и 1 чёрным шарами достают наугад 1 шар:

$A$ = \{Извлечённый шар красный\} = $\varnothing$

$B$ = \{Извлечённый шар чёрный или белый\} = $\Omega$

\subsubsection*{2. Геометрическое определение вероятности}

Геометрическое определение вероятности обобщает классическое на случай, когда $\Omega$ является бесконечным множеством элементарных исходов.

Пусть $A \subseteq \mathbb{R}^n$. Через $\mu(A)$ будем обозначать меру мн-ва $A$
\begin{enumerate}[topsep=0pt, leftmargin=2pt, noitemsep]
	\item [] $n=1$, то $\mu$ -- длина
	
	\item [] $n=2$, то $\mu$ -- площадь
	
	\item [] $n=3$, то $\mu$ -- объём. и т.д.
\end{enumerate}

1) $\Omega \subseteq \mathbb{R}^n, ~\mu(\Omega)<\infty$

2) $A\subseteq\Omega$

3) Возможность принадлежности некоторого исхода СЭ событию пропорционально мере события и не зависит ни от формы события, ни от его расположения внутри $\Omega$.

\OPR Вероятностью события $A$ наз. число $P(A) = \tfrac{\mu(A)}{\mu(\Omega)}$

\ZAM 1) Очевидно для геометр. опред. вероятности остаются в силе св-ва 1-3 класс. вероятности
2) Недостаток: не учитывает возможность того, что некоторые области внутри $\Omega$ окажутся более предпочтительными, чем другие -- в таком случае полученный результат будет неадекватным.

\subsubsection*{2. Статистическое определение вероятности}

Пусть

1) случайный эксперимент повторяется $n$ раз

2) при этом событие $A$ произошло $n_A$ раз

\OPR Вероятностью события $A$ наз. эмпирический (полученный опытным путём) предел отношения $\tfrac{n_A}{n}$ при $n\rightarrow\infty$

\ZAM 1) Очевидно для статистического. опред. вероятности остаются в силе св-ва 1-3 класс. вероятности

\ZAM Недостатки: 

1) опыт не может быть повторён бесконечное число раз 

2) такое определение не даёт достаточной основы для дальнейшего развития матем. теории

\clearpage
